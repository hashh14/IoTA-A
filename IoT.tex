\section{Introduction:}
Internet of Things (IoT) technology has a wide variety of applications. Based on the different areas of application of Internet of Things, systems are designed accordingly. The architecture of IoT depends heavily upon its functionality and implementation in different sectors. Regardless, there is still a basic process flow based on which IoT is built.

\section{Architecture:}
There are 4 layers present, and they are divided as shown:
Sensing Layer, Network Layer, Data processing Layer, and Application Layer.

\subsection{Sensing Layer:}
It is also known as the perception layer. This layer consists of sensors, actuators, and other perception devices. These sensors or actuators receive data, which are then processed and emitted over networks. It obtains useful information, and then transmits it to the network layer through the network access devices such as WSN gateways. This layer consists of integrated hardware for the acquisition of data and perception. The most popular sensing technologies are RFID, Camera, sensors, barcode and others.

\subsection{Network Layer:}
The main function of this layer is information transmission across the network. Internet/Network gateways, Data Acquisition System (DAS) are present in this layer. DAS performs data aggregation and conversion function, which is collecting and aggregating heterogeneous data, and then converting analogue data of sensors to digital data. Advanced gateways which mainly opens up connection between Sensor networks and Internet also performs many basic gateway functionalities like protection from malware, denial of unauthorized access and preventing the breach of confidential data.

\subsection{Data processing Layer:}
This layer exists between the application and network layer. This is the main processing unit. Here data is analysed and pre-processed before sending it to a data centre from where developers can concentrate on the application development process. Here edge analytics comes into picture. Edge is the hardware and software gateway that analyses and pre-processes the data before transferring it to the cloud. This layer is also responsible for ensuring interoperability, scalability, abstraction and providing service for customers.

\subsection{Application Layer:}
This is last layer of 4 stages of IoT architecture. It acts as an interface between the Internet of Things and users The application layer is used to build applications to satisfy the needs of the customer. The layer includes key technologies such as distributed computing, intelligent processing of massive information, information findings. It caters to the needs of a variety of applications like agriculture, healthcare, aerospace, farming, etc. 


\section{Applications:}

\subsection{Agriculture:}
Internet of Things can be of great use in the field of agriculture. It can be helpful in monitoring growth of medicinal plants. These plants are fitted with RFID tags and sensors. When there is a drastic or unexpected change in the growth of plant due to temperature / humidity, the sensors sense this and the RFID tags send the information to the reader and are shared across the internet. The farmer or scientist can access this information from a remote place and take necessary actions.

\subsection{Healthcare:}
IOT plays a crucial role in healthcare. It can be used in many ways such as tracking the number of patients in a hospital, identifying the right patient for the right medicine and monitoring a patient’s health conditions from a remote place which is known as Telemedicine. This includes providing treatment, diagnosis and treatment. Ambient assisted living provides technical systems for elderly people who are alone at home and need to be monitored. The patient’s health status is periodically sensed using RFID and sensors. The doctor from a remote location provides medical assistance based on the information received.

Although IoT offers many advantages and uses through its application, there are still a few challenges that are involved. The main challenges are privacy, reliability, data confidentiality and security. For example, A vehicle attached with RFID tag leads to lack of privacy for the passenger in the vehicle. IOT in healthcare can also lead to dangerous consequences such as the data present in the system can be changed by an intruder, leading to discrepancies in patient data.

